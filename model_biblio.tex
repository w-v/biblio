\documentclass[11pt]{sdm}
\usepackage{graphicx}
\usepackage[
backend=biber,
style=alphabetic,
sorting=ynt
]{biblatex}
\addbibresource{bib.bib}

%numeroter les pages
\pagestyle{plain}


\title{Synthesising Motion Variations in Human Crowds}
\author{Robin \textsc{Adili}}
\supervisorOne{Ludovic \textsc{Hoyet}}
\supervisorTwo{Anne-Hélène \textsc{Olivier}}
\supervisorThree{Julien \textsc{Pettré}}
\team{MIMETIC / RAINBOW}
%One of:
% ens-Rennes  esir    insa-rennes   rennes1  
% enssat    logoUbs   tsupelec
%here rennes1 for example
\school{rennes1}


% the domain should be one or two of:
% Technology for Human Learning 
% Artificial Intelligence 
% Computer Arithmetic
% Hardware Architecture
% Automatic Control Engineering
% Bioinformatics 
% Biotechnology
% Computational Complexity 
% Computational Engineering, Finance, and Science
% Computational Geometry 
% Computation and Language 
% Cryptography and Security 
% Computer Vision and Pattern Recognition
% Computers and Society 
% Databases 
% Distributed, Parallel, and Cluster Computing 
% Digital Libraries
% Discrete Mathematics 
% Data Structures and Algorithms 
% Embedded Systems 
% Emerging Technologies 
% Formal Languages and Automata Theory 
% General Literature 
% Graphics 
% Computer Science and Game Theory 
% Human-Computer Interaction 
% Computer Aided Engineering 
% Medical Imaging 
% Information Retrieval 
% Information Theory 
% Ubiquitous Computing 
% Machine Learning
% Logic in Computer Science 
% Multiagent Systems 
% Mobile Computing
% Multimedia
% Modeling and Simulation 
% Mathematical Software 
% Numerical Analysis 
% Neural and Evolutionary Computing 
% Networking and Internet Architecture 
% Operating Systems 
% Performance 
% Programming Languages 
% Robotics 
% Operations Research
% Symbolic Computation 
% Sound
% Software Engineering 
% Social and Information Networks 
% Systems and Control 
% Image Processing 
% Signal and Image Processing 
% Document and Text Processing
% Web
\domain{Domain: Graphics}

%write your abstract here
\abstract{Virtual human crowds are regularly featured in movies and video games. With a large number of virtual characters each behaving in their own way, they make spectacular scenes. The more diverse the characters and their behaviors are, the more realistic the virtual crowd will be. Hence, creating virtual crowds is a trade-off between the cost associated with acquiring more of assets, namely more virtual characters with their animations, and realism. Our focus is on character animation. To alleviate the cost of capturing more varied animations, a possible approach is to synthesize variations of existing captures. In addition, variations can be used only where they are the most likely to be noticed, only on characters close to the camera for example. To prepare and inform our future works on this approach, we present in this report a study of the literature regarding motion variation and style synthesis, and the perception of variety in virtual human crowds.}



\begin{document}
\maketitle

\section{Introduction}

\subsection{Crowds in human societies and entertainment}
In human societies, crowds are a common phenomenon. People voluntarily gather in large quantities for political events and entertainment. Crowds also spontaneously appear where population density is high and people are just travelling around. 
Because of the sheer number of individuals, each behaving in their own personal way, and potentially interacting between each other, these scenes are complex and present a lot of information.
As such, they make compelling content for visual entertainment.
(examples of movies and video games)

\subsection{Virtual crowds}
(Why)
What makes crowds appealing also makes them difficult to recreate. Until recent decades, filmmakers needed to hire and coordinate hundreds if not thousands of extras to capture a scene featuring a large crowd. This represented a prohibitive cost for small productions, and effectively restricted their apparition to large budget movies.
With the development of computer graphics and visual effects, crowds can nowadays be generated and composited into movies.
(How)
Animating and rendering such a large number of characters requires dedicated techniques.
(trajectories+animation)
Capturing or creating the appearance and animation of every single character would defeat the purpose of simulating the crowd. It would ultimately be comparable to filming a real one.
These virtual crowds are usually created from a smaller set of characters. Some of the virtual humans are identical.

\subsection{Our problem}
The smaller the set of original characters, the more uniform the crowd will look
Uniform crowds are unrealistic, in real life people are different, in their appearance and in the way they move.
Variation is needed, but costly.
We focus on animation. Usually done with motion capture
Either extend the set of captured motions. (explain why mocap is costly). 
Or synthesize motion variations from the set of captured motions. Still costly in processing time for generation, and memory.

\subsection{Proposed approach}
We are interested in synthesis, less costly compared to more mocap
The goal is to create motions that will make characters seem more unique
Humans do not perceive everything they see (reference). 
While creating varied features for virtual humans in a crowd, we might introduce some that will not be noticed by most people, and could have been omitted.
The goal of this work is to find out the variables that can determine the perception of variety in motion of characters of a crowd.
With knowledge of these criteria, variety can be introduced where it is most needed to improve realism
(can be a guideline for studios implementing crowd systems with limited budget)
(examples of criteria)
where on the screen ? where in the scene ? where on the characters ? what proportion of characters ?

(why is it difficult ?) 
(why the proposed approach is difficult)
Perception is not deterministic, nor easily predicted (optical illusions)(reference)
(exemple)
We can make assumptions, but the best way to know is to evaluate
Synthesizing convincing human motion is hard. Human motion is incredibly complex (reference) (part of the problem ?)
parmi tout l'ensemble des DOF des articulation, seule une petite partie sera naturelle, et encore une plus petite partie sera appealing
comment est-ce qy'on passe à l'échelle, pour deux trois mille personne ça sera plus difficile 
où est-ce qu'on met la variation 

\subsection{Outline of the bibliographic study}
To provide an answer, we need to survey the litterature regarding human motion synthesis, and perception of variety in crowds

\subsubsection{Human motion variation synthesis}
Human motion synthesis is used in other problems : generating transitions between motions (motion graphs), generating physically based animations, based on environmental interactions
Motion graphs won't do because they don't modify the animations, only put parts together, in our case, that would still produce a certain uniformity of motions
Here we focus on synthesizing variations of a certain motion 
The goal is to have a good understanding of existing method that could be applied to create animations of different characters so that they feel as different, personal, and unique as real people, but don't necessitate a lot of motion capture with a lot of actors.
Questions motivating the search
What type of motion can synthesized ? (style and variations)
How much input data ? From the same actor ? How well does it transfer to characters of different morphology ?
How easy is it to implement ? Performance ? Quality (artifact free, realistic) ? 
Can it be applied to our problem of generating variations in crowds 

\subsubsection{Character variety in crowds}
Is there proof that variety in crowds improves realism ?
What type of variety has been studied (appearance and animation)
What are the most frequently perceived features
Which is the most noticeable ?
Is there proof that variety in motion is perceptible ?

\section{Synthesizing variation in human motion}

(why)
Synthesizing human variation motion instead of animating it by hand or capturing it can alleviate the workloads of actors and artists.

human motion is complex (reference). Physical, psychological determinants
A synthetic motion should appear physically realistic, joints cannot be reversed etc.
Ideally, it should be recognised as produced by a real person

An instance of human motion can be modeled as the output of a function of multiple variables.
Firstly motion is performed by humans as an action, for example a walk.
Secondly, a given action can be carried out with different styles, conveying different meanings, for example a walk can be sad, agressive, or happy.
These two dimensions do not account for the personal gait of a particular individual, or the way he will choose to perform these actions. They can be thought of as neutral. 
Indivisuals perform these motions with variations specific to themselves. (inter personal)
The last variable is the variations obtained when someone repeats a particular style of action. (intra personal)

In the first part, we will discuss articles that studied style synthesis. Then we will focus on contributions to synthesis of variations

\subsection{Human motion style synthesis}

Style can be thought of as the way to carry out an action. It can carry information about emotions of the actor, or its motivations.
(style can be parameter like speed or stride distance)
The litterature comprises studies of various ways to stylize an existing motion. Synthesis is usually based on a database of motions including motions of the source and target style but with different characteristics|features than the input motion.
These methods are called style transfer, because they effectively transfer a style to a new motion having different characteristics. 
Independent of personal variations, everybody can walk angrily, it won't be exactly the same walk, but it will carry the same meaning
(some of them are real time)
(talk about the technical similarities, time and space warp)
(introduce time warping and space warping Witkin et al. \cite{Witkin:1995:MW:218380.218422})

\subsubsection{Style tranfer to new content}

Hsu et al. \cite{hsu2005style} proposed a method to transfer style to long motions having a different content. In this context, content is understood as different actions being part of the same motion, like making a turn or stepping over an obstacle.
The database is made of neutral motions, without any particular actions.
Because of the potential difference in length, motions need to be time warped. Because style is different, individual frames are spatially different and need space alignment
Synthesis is done by minimizing the difference between target motion and warped source motion, additional terms regulate smoothness and monotony of warping. (equation ?)
Space warping is linear, and invariant in time after time warping, thus LTI linear time invariant [ref]

\subsubsection{Style transfer to new actor}

Urtasun et al. \cite{urtasun2004style} studied how to transfer style between motions of different actors, while retaining their personnal features.
Here, style is a continuous parameter of the motion. The speed of a walk, the length of a jump.
The authors capture motions of actors for various values of the parameters
This contribution allows interpolating the style parameter to synthesize a new motion of a known actor 
Style parameters can also be extrapolated to synthesize the motion of a new actor, not present in the database
The method uses PCA to compute a common parametric space between motions.
The main contribution is that input motions are expressed in PCA space using approximation, without having to recompute the SVD on the whole database comprising the new motion.
For a new actor performing an action with a given style parameter, the eigendecomposition approximated by its is projection onto a known motion of a different actor with the same parameter value.
Extrapolation is computed by comparing weights of source and input motions and reapplying the differences to target motion
(mahalanobis distance ?)
(equation ?)


Min et al. \cite{min2010synthesis} \\
Parameter space for style and actor, in the form of motion = f(style, actor).\\
generative model : interpolate to get new styles/actors\\
also get new styles for existing actors : given a known actor performing a new style, get other known actors performing that new style\\
new actors for existing styles : given new actor performing known style, get that new actor performing another known style, that has not been recorded\\
multilinear data analysis on PCA weights \\
(talk about cost function)(still need to work on technicalities)

\subsection{Interpersonal and intrapersonal motion variation synthesis}

Variations in human motion across people performing an action with a particular style
Comes from psychological and sociological factors \cite{multon:tel-00441143}. Can come from the morphology of the actor.
O'sullivan \cite{o2009variety} suggests that differences between character and actor morphology can produce less realistic motions
variations across performances of the same person : important in case of cyclic motions : every cycle should be different
(harris et al. \cite{harris98} les variations viennent des erreurs de signaux nerveux, mais ne sont pas entièrement aléatoire
l'erreur dépend du mouvement, un mouvement rapide sera moins fidèlement reproduit)


The naive approach, formulated by Perlin \cite{perlin} consists in applying random variations to an existing animation. However, random variations are not coherent with the motion, or the constraints of the human motor system. 
Variations appear on sollicited muscles. 
A more successful approach is to try and learn the distribution of the variations for every joint from a set of examples. Then synthesis procedes by sampling from that distribution to get new variations
(same as with styles) problem is that examples can be of different length, and alignment with respect to time can be necessary

\subsubsection{Models for variation and style synthesis}
Ma et al. \cite{ma2010modeling}\\
Style is continuous parameter, measurable parameter as in Urtasun et al. \cite{urtasun2004style}. Here stride length, double support duration for a walk\\
User can select style parameters and get infinitely many variations of those style parameters.\\
Variations are style dependent. Variation is sampled from examples matching user chosen style parameters\\
Divides an animation in partial animations between keyframes, partial motions from different styles are then compared, used as time warping\\
Divides skeleton in joint groups and keeps the joint with the most variance as the source of variations\\
(more on BN)

\subsubsection{Models for variation synthesis}
Lau et al. \cite{lau2009modeling}\\
is related to Ma et al., but style is not modeled\\
BN too, but on poses\\
no time warping at all\\
defines variation as difference in repeated performance of same action with same style by same person\\
targets variations in repeated motions like cyclic motions, but can be applied to variation across characters \\
prior network and trans network \\
kind of a recursion, prior gives initial values, then trans gives next from 2 last and repeat\\
if movement is to long, initialization is lost\\

Zhou et al. \cite{zhou2014human}
Gaussian process : fitting Gaussians to variance of database to get probability distribution\\
ancestor joints from previous 2 frames used\\
DOFs of a joint are linked via linear combination : better model joint dependencies\\

\section{Variety in Crowds of Virtual Humans}

trajectories + animations
Capture a limited set
Extend it 
Why ?
realism
How to extend ?
variety
what kind ?
appearance and animation
How ?
only necessary variations, that serve realism
what are those ?\\
Hoyet et al. 2012 \cite{hoyet2012push}\\
Variations in motion intensity are likely to be perceived\\
Thus using the same animation for all characters of a virtual crowd is likely to impact realism\\


\subsection{Performance challenges}
What are challenges related to performance of virtual human crowds \\
Cannot have individual animation for 10k characters : not enough memory, too much geometry \\
Principle of a LOD : why use the same amount of compute on close and far away chars \\
Dobbyn et al. \cite{dobbyn2005geopostors} :\\
LOD in rendering \\
impostors instead of geometry\\
Kulpa et al.\cite{kulpa2011imperceptible} :\\
LOD in trajectory computation for collision avoidance\\
Shows that perception of certain details is limited when the viewer gets further away

\subsection{Perception of variety in appearance}
Problem is similar to ours : how to use variety efficiently\\
Although studies mostly appearance\\
Evaluation based on clone spotting\\
McDonnell et al. 2008 \cite{mcdonnell2008clone} asks subjects to click on clones\\
Compares perception of variations in motion and appearance \\
McDonnell et al. 2009 \cite{mcdonnell2009eye} uses saliency to evaluate clone detection\\
Uses body part as a criterion for evaluating if variation is perceived (we plan to use it but for motion)\\  

\subsection{Perception of variety in motion}
Hoyet et al. in 2016 \cite{hoyet2016perceptual}\\
Introduces variations based on interactions\\
Scenes are perceived as more natural as a result\\
Hoyet et al. in 2013 \cite{hoyet2013evaluating}\\
Variation in motion between actors is important as it is well perceived\\
Some motions are more distinctive between actors than others\\
The same actor might be recognisable when he plays different characters\\
Some ways motions are performed are more appealing than others\\

Prazak et al.\cite{pravzak2011perceiving}\\
Large DB of motions\\
Evaluates naturalness of a crowd when the number of source animations vary\\
Subjects always have a golden reference where every animation is captured available on another monitor\\
Factors in the character morphology in relation to the actor morphology of its assigned animation\\ 


\section{Conclusion}
Practical ways this knowledge can be applied to our problem\\
Hypotheses on the future experimental results :\\
Based on existing experiments studied here (or after personal analysis), we can assume that\\
Test subjects will probably notice uniformity of character animation less and less as the character are farther away in the crowd \cite{kulpa2011imperceptible} \cite{mcdonnell2008clone}\\
Some body parts might get occluded more often than others in a crowd setting, thus we can expect the uniformity of their motion to be less noticed as well \cite{mcdonnell2009eye}\\


\newpage
\printbibliography
\end{document}

%%% Local Variables:
%%% mode: latex
%%% TeX-master: t
%%% End:
